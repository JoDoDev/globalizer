\documentclass[12pt]{article}

\usepackage[utf8]{inputenc}
\usepackage[T1]{fontenc}

% clickable links in pdf
\usepackage{hyperref}

% make title variable available
\usepackage{titling}

% create own header footer
\usepackage{fancyhdr}

% images
\usepackage{graphicx}

% grafics powerfull and hard
\usepackage{tikz}

% character spacing to fill line
\usepackage{microtype}

% image placement with [H]
\usepackage{float}

% fancy quotes
\usepackage{epigraph}

% extra symbols
\usepackage{textcomp}

% display code
\usepackage{listings}

% icons
\usepackage{fontawesome}

% set dimmensions of page
\usepackage[footskip=80pt, headheight=15pt]{geometry}

% tables
\usepackage{tabularx}
\usepackage{makecell}

% make code copyable
\lstset{
	upquote=true,
	columns=fullflexible,
	literate={*}{{\char42}}1
	{-}{{\char45}}1
}

\newsavebox{\picbox}

\graphicspath{ {./images/} }

% command for rounded corners
\newcommand{\cutpic}[3]{
	\savebox{\picbox}{\includegraphics[width=#2]{#3}}
	\tikz\node [draw, rounded corners=#1, line width=4pt,
	color=white, minimum width=\wd\picbox,
	minimum height=\ht\picbox, path picture={
		\node at (path picture bounding box.center) {
			\usebox{\picbox}};
	}] {};}


	\newcommand{\apiSpecToblerone}[4]{
	    \begin{table}[H]
	      \begin{tabularx}{\textwidth}{|l|X|}
	        \hline
	        Typ & #1  \\ \hline
	        Name & #2  \\ \hline
	        Content & #3  \\ \hline
	        Beschreibung & #4  \\ \hline
	      \end{tabularx}
	    \end{table}
	}
