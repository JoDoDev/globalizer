\documentclass[12pt]{article}

\usepackage[utf8]{inputenc}
\usepackage[T1]{fontenc}

% clickable links in pdf
\usepackage{hyperref}

% make title variable available
\usepackage{titling}

% create own header footer
\usepackage{fancyhdr}

% images
\usepackage{graphicx}

% grafics powerfull and hard
\usepackage{tikz}

% character spacing to fill line
\usepackage{microtype}

% image placement with [H]
\usepackage{float}

% fancy quotes
\usepackage{epigraph}

% extra symbols
\usepackage{textcomp}

% display code
\usepackage{listings}

% icons
\usepackage{fontawesome}

% set dimmensions of page
\usepackage[footskip=80pt, headheight=15pt]{geometry}

% tables
\usepackage{tabularx}
\usepackage{makecell}

% make code copyable
\lstset{
	upquote=true,
	columns=fullflexible,
	literate={*}{{\char42}}1
	{-}{{\char45}}1
}

\newsavebox{\picbox}

\graphicspath{ {./images/} }

% command for rounded corners
\newcommand{\cutpic}[3]{
	\savebox{\picbox}{\includegraphics[width=#2]{#3}}
	\tikz\node [draw, rounded corners=#1, line width=4pt,
	color=white, minimum width=\wd\picbox,
	minimum height=\ht\picbox, path picture={
		\node at (path picture bounding box.center) {
			\usebox{\picbox}};
	}] {};}


	\newlength{\textundbildtextheight}

	\newcommand{\textundbild}[1]{
	\vfill
	\begin{center}
	\includegraphics[width=\textwidth,keepaspectratio=true,height=\textheight-\the\textundbildtextheight]{#1}
	\end{center}
	\vfill
	}


\begin{document}
	\title{
	\Huge
	\textbf{Globalizer} \\
	\vspace{0.2cm}
	\LARGE
	Detailkonzpet
}

\date{22.11.2018}

\author{
	Koller, Jonas\\
	\texttt{jonas.koller@gmx.ch} \\
	Wolfisberg, Donato \\
	\texttt{donato.wolfisberg@gmail.com}
}

\pagestyle{fancy}
\fancyhf{}
\lhead{BBZW Sursee Rötheli Manfred}
\rhead{Globalizer Detailkonzept}
\lfoot{Jonas Koller \& \\ Donato Wolfisberg}
\cfoot{\thedate}
\rfoot{\thepage}

\renewcommand{\headrulewidth}{1pt}
\renewcommand{\footrulewidth}{1pt}

\renewcommand{\contentsname}{Inhalt}

\begin{titlepage}
  \pagenumbering{gobble}

  \begin{center}
    \vspace*{-2cm}
    \cutpic{0.8cm}{4cm}{logo.jpg}

    \thetitle

    \vspace{2cm}

    \textbf{\theauthor}

    \vspace{1.5cm}

    \thedate
  \end{center}

  \vfill

  \begin{figure}[H]
    \makebox[\linewidth]{
      \includegraphics[width=1.3\linewidth]{globe.jpg}
    }
    \vspace*{-5cm}
  \end{figure}
\end{titlepage}

\newpage
\pagenumbering{Roman}


	\begin{center}
		\makebox[\textwidth]{\includegraphics[width=\paperwidth]{nightsky.jpg}}
	\end{center}

	\section{Einleitung}
	Dieses Dokument behandelt das Detailkonzept des Projekts Globalizer. Ziel und Zweck des Dokuments ist es, die fachlichen Anforderungen technisch zu umschreiben, damit diese später implementiert werden können. Zudem soll das Gesamtbild unserer Applikation klar gemacht werden und die Komponenten, sowie der Datenaustausch genau beschrieben werden. Es werden auch all unsere Technologieentscheide vermerkt und begründet. Zusammengefasst soll dieses Dokument also eine klare, wie auch detailierte Übersicht zu den technischen Aspekten unseres Projekts sein.

	\newpage

	\section{Allgemeine Informationen}
	Hier folgt eine kurze Auflistung der Informationen zu diesem Dokument, dem Ent\-wicklungsteam und dem aktuellen Stand.

	\subsection{Projektmitarbeiter}
	\begin{table}[h]
		\begin{tabularx}{\textwidth}{|l|l|X|l|}
			\hline
			\textbf{Name} & \textbf{Vorname}  & \textbf{E-Mail}                & \textbf{Funktion}     \\ \hline
			Koller        & Jonas             & jonas.koller@gmx.ch            & Projektleiter         \\ \hline
			Wolfisberg    & Donato            & donato.wolfisberg@gmail.com    & Entwickler            \\ \hline
			Gian          & Ott               & gian\_ott@sluz.ch              & Prüfer                \\ \hline
			Manuel        & Troxler           & manuel\_troxler@sluz.ch        & Prüfer                \\ \hline
		\end{tabularx}%
	\end{table}

	\subsection{Änderungskontrolle}
	\begin{table}[h]
		\begin{tabularx}{\textwidth}{|l|l|l|X|}
			\hline
			\textbf{Version} & \textbf{Datum} & \textbf{Ausführende Stelle} & \textbf{Bemerkung}                     \\ \hline
			1                & 21.09.2018     & Projektteam                 & \makecell[l]{Erste Version des Dokuments \\ erstellt}  \\
			2                & 23.09.2018     & Projektteam                 & \makecell[l]{Gesamtüberblick erstellt}  \\
			3                & 25.09.2018     & Projektteam                 & \makecell[l]{Zielkatalog erstellt}  \\
			4                & 27.09.2018     & Projektteam                 & \makecell[l]{Abschliessende Arbeiten}  \\
			\hline
		\end{tabularx}
	\end{table}

	\subsection{Prüfung}
	\begin{table}[h]
		\begin{tabularx}{\textwidth}{|l|l|X|}
			\hline
			\textbf{Version} & \textbf{Datum} & \textbf{Ausführende Stelle}     \\ \hline
			4                 & 27.09.2018    & Gian Ott                        \\ \hline
		\end{tabularx}
	\end{table}

	\newpage
	\tableofcontents
	\newpage

	\pagenumbering{arabic}


	\section{Systemkomponenten}
	Das Projekt verwendet einen klassische Client - Server Architektur. Somit ergeben sich bei uns drei Hauptkomponenten. Dies ist das Backend, das Frontend und die Datenbank. Wir werden folgend beschreiben, wie diese genau aufgebaut sind.
	\subsection{Frontend}
	In der Fontend-Komponente soll die Schnittstelle zwischen Endbenutzer und Backend-System implementiert werden. Wir werden dies mit einer Weboberfläche machen. Diese Komponente soll unabhängig vom Backend-System auf einem CDN gehostet werden können. Diese Masnahme verringert die Wartenzeiten des Endnutzers deutlich.
	\subsection{Backend}
	Unser Backend-System soll die Schnittstelle zwischen den Daten und dem Frontend abbilden. Hier werden die rohen Daten von der Datenbank aufbereitet, berechnungen durchgeführt und die Security-Aspekte grösstenteils abgedeckt. Die Authentifizierung, Autorisierung und Überprüfung der übermittelten Daten soll hier stattfinden. Das Backend soll möglichst klein gehalten werden, damit der Wartungsaufwand minimal bleibt. Es soll so gebaut werden, dass es gut skalierbar verwendet werden kann. Aspekte wie Load-Balancing und Upscaling sollen vom Hoster übernommen werden.
	\subsection{Datenbank}
	Die Datenbank bildet den dritten Teil unseres Systems. Hier soll eine InMemory-Datenbank verwendet werden. Diese brauch weniger Leistung und sorgt für zusätzliche Geschwindigkeit beim System. Uns ist bewusst, dass die Datenbank ihren aktuellen Stand bei einem Systemabsturz verlieren kann. Da es sich jedoch um einen flüchtigen Chat handelt, welcher nicht zu 100\% sicher persistiert werden muss, kann eine InMemory-DB verwendet werden.

	\newpage

	\subsection{Zonen-Übersicht}
	Auf der nachfolgenden Grafik kann unsere Architektur im groben ausgelesen werden. Das System ist in die drei "Komponenten-Zonen" unterteilt. Diese sollen wenn möglich unabhängig von den anderen deploybar sein. Dies ermöglicht es uns später auch, unsere Deployment-Zyklen zu vereinfachen. Im Backend-System ist zusätzlich auch der LoadBalancer eingezeichnet, welcher vom Hoster übernommen wird.\newline

	\noindent
	\includegraphics[width=\linewidth]{system-components.png}

	\section{Design-Pattern}
	Da werder unser Backend, noch unser Frontend objektoriertiert ist, können keine Klassendiagramme erstellt werden. OOP-Eigenschaften wie Vererbungen und beziehungen werden bei diesem Projekt auch kaum benutzt. Aus diesem Grund verzichten wir auf diese Diagramme. Wir verwenden anstelle dessen aber ein UML-Aktivitätsdiagramm, um unsere Abläufe zu visualisieren. Wir venwenden in diesem Projekt keine Domain-Driven-Design, Test-Driven-Design oder ähnliches, da unser Projekt zu klein ist, damit dies den Aufwand wert wäre.

	\subsection{Allgemeine Architektur-Pattern}
	Wir wollen die Kommunikation zwischen Frontend und Backend möglich dynamisch und asynchron halten, damit weder das Front- noch das Backend lange Wartenzeiten aufweist oder "einfriert". Da die Kommunikation selbst aufgrund der Technologien nur schwer ganz asynchron umsetzbar ist, verwenden wir das in Angular häufig angewanndte "pseudo Asynchronität". Die Anfragen an sich verlaufen synchron. Angular lässt aber zu, dass das Frontend wärenddessen nicht blockiert wird und arbeitet bis zur Antwort den restlichen Befehlsstack ab und hört auf Events. Im Backend läuft das auch so. Somit blockieren werder das Back- noch das Frontend und der Nutzer hat ein vielfach besseres Erlebnis beim verwenden der Applikation. Sowohl verwenden auch an beiden Orten das "Observer"-Pattern. Dies wird durch Angular und NodeJS unterstützt, in dem wir die "rxjs"-Library verwenden. Durch die einbindung dieser vereinfacht sich das Behandeln von Events und das reaktive Programmieren.

	\subsection{Codestyle und Codequalität}
	Damit unser Projekt auch über lange Zeit verwendbar, Wartbar und Erweiterbar ist, werden wir einige Programmierprinzipien anwenden, welche sicherstellen, dass unser Code langzeitig sauber bleibt. Dies sind folgende:
	\begin{itemize}
		\item \faCode~ Cleancode-Prinzip
		\item \faInstitution~ SOLID-Prinzip
		\item \faUsers~ Pair-Programming-Prinzip
		\item \faGit~ Feature-Branches und Pull-Requests
	\end{itemize}
	Falls einer der Begriffe unbekannt sein sollte, kann eine Erklärung zu diesem im Internet gefunden werden.\newline
	Wir verwenden Github als Versionskontrolle (mehr dazu bei den Technologieentscheiden). Damit wir die oben genannten Prinzipien einhalten, setzen wir \href{https://www.codefactor.io}{Codefactory} ein. Zudem verwenden wir auch \href{https://palantir.github.io/tslint/}{TSLint}, eine Clientseitige Library zur Überprüfung vom Codestyle.\newline
	Desweitern verwenden wir einen Continuous Build System und ein Continuous Testing System, welches aktiviert wird bei einem "Push" ins Github Repository. Dazu auch später mehr.

	\subsection{Architektur Backend}
	Das Backend wird in NodeJS erstellt. Es nutzt Websockets zur Kommunikation mit dem Frontend. Das Websocket-Protokoll ist auf TCP basiert und erlaubt bidirektionale Verbindungen zwischen Server und Client. Das Backend ist offen für neue Verbindungen. Wenn sich ein neuer Benutzer anmeldet, registriert dieser sich beim Backend. Dieses hört danach auf Anfragen des Frontends. Im gegenzug sendet das Backend neue Nachrichten selbst an das Frontend. Durch dieses "Bidirectional Message Pattern" können wir einiges an Bandbreite einsparen, da wir nicht immer wieder Fetch-Anfragen senden müssen um den Client auf dem aktuellen Stand zu behalten. Wir konnten somit das "Fetch"-Antipattern umgehen, da dieses für eine Chat-Applikation ungeeignet ist. In der folgenden Grafik kann ausgelesen werden, wie die der Ablauf bei Anfragen an den Server aussehen.

	//TODO: Bild Backend aufrufe

	\subsection{Architektur Frontend}
	Unser Frontend wird mit Angular (Version 7) implementiert. Wieso genau wir uns für diese Technologie entschieden haben wird später erläutert. Dazu wird das Google Angular Material verwendet, damit wir dank dieser Library nicht alle UI-Komponenten selbst erstellen und designen müssen, sondern auf diese UI-Library zurückgreifen können (diese Library ist ähnlich wie Twitters Bootstrap). Das Frontend sendet auch via Websocket-Protokoll Nachrichten an das Backend.

	\section{Persistenz}
	Unsere Applikation verwendet wie bereits oben beschrieben eine InMemory Datenbank. Diese wird eine NoSQL Datenbank sein (NoSQL im Sinne von kein SQL verwendend und nicht darauf basierend). Sie basiert auf einem JSON-Datenmodel. Dies da wir in JavaScript programmieren werden und die Daten verwenden können, ohne diese zuerst parsen zu müssen. Die InMemory-DB wird durch Vanilla-JavaScript ermöglicht und braucht keine weiter Library. Ein konkretes Datenmodel gibt es nicht aufgrund von dem Fakt, dass wir dynamisch Daten speichen und entfernen und das genaue Datenmodel so entsteht.

	\section{Detailierte Komponenten und Schnittstellen}
	Folgend werden alle Komponenten genau beschrieben und deren Schnittstellen definiert.

	\subsection{Backend}
	Die Backendkomponente bildet das "Hirn" unserer Applikation. Hier werden sowohl alle wichtigen Daten gespeichert (in der InMemory-DB), als auch alle wichtigen Impulse verwendet (Websockets). Ziel des Backends ist es eine möglichst hohe flexibilität und Wartbarkeit mit einer möglichst hohen Performance zu kombinieren. Dies wird durch die sehr junge Backend-Sprache NodeJS ermöglicht.

	\subsection{Funktionale Anforderungen}
	Das Backend ist betroffen von 3 obligatorischen funktionalen Anforderungen. Diese sind:
	\begin{itemize}
		\item \faLock~ USER\_STANDARD\_LOGIN
		\item \faSend~ NACHRICHT\_SENDEN
		\item \faEnvelope~ NACHRICHT\_EMPFANGEN
	\end{itemize}
	Was diese Anforderungen konkret bedeuten kann im Dokument "Anfoderungsspezifikationen" nachgelesen werden. Dort sind diese inklusiv Use-Case-Diagramm dargestellt.

	\subsection{Schnittstellen}
	//TODO: Schnittstellen beschreiben

	\subsection{Presistenz}
	Das Backend übernimmt den Hauptteil der Persistenz der ganzen Applikation. Hier werden alle Nachrichten und Benutzer in die InMemory-DB gespeichert und der Zustand der ganzen Applikation.
	Die Entitäten sehen wie gefolgt aus:
	//TODO: Bilder Entitäten

	\subsection{Frontend}
	Die Frontendkomponente bildet die Schnittstelle zwischen Endbenutzer und System. Hier werden die Daten aus dem Backend grafisch dargestellt. Wichtig an dieser Komponente ist, dass das Aussehen dieser modern und stylisch ist, damit der Nutzer ein tolles Erlebnis auf der Seite hat. Auch Geschwindigkeit ist hier ein wichtiges Thema, weshalb Angular 7 eine sehr gut passende Technologie ist. Die Funktionsweise des Frontends kann aus diesen UML-Diagramm ausgelesen werden.

	//TODO: UML Flussdiagramm Frontend

	\subsection{Funktionale Anforderungen}
	Das Frontend ist betroffen von 3 obligatorischen funktionalen Anforderungen. Diese sind:
	\begin{itemize}
		\item \faLock~ USER\_STANDARD\_LOGIN
		\item \faSend~ NACHRICHT\_SENDEN
		\item \faEnvelope~ NACHRICHT\_EMPFANGEN
	\end{itemize}
	Was diese Anforderungen konkret bedeuten kann im Dokument "Anfoderungsspezifikationen" nachgelesen werden. Dort sind diese inklusiv Use-Case-Diagramm dargestellt.

	\subsection{Schnittstellen}
	Die Schnittstelle zwischen Benutzer und Frontend ist rein visuell. Diese soll später so aussehen:

	\subsubsection{Android frame}
	\includegraphics[width=\textwidth]{mock-chat.png}
	\subsubsection{Desktop Chrome Login}
	\includegraphics[width=\textwidth]{frame-desktop-login.png}
	\subsubsection{Desktop Chrome Chat}
	\includegraphics[width=\textwidth]{frame-desktop-chat.png}

	\subsection{Presistenz}
	Das Frontend hat nur einen ganz minimalen Teil an Persistenz. Dazu wird der durch HTML 5 neu eingeführte LocalStorage eingesetzt. Es wird nach dem Login eines Benutzers dessen ID gesichert, damit später ein Sessionlogin durchgeführt werden kann und der Nutzer sich nicht erneut anmelden muss. Da diese Daten aber auf dem Client selbst gespeichert sind, müssen sie mit vorsicht behandelt werden.

	\section{}

#Technologieentscheide
	//TODO: Wieso Github
	//TODO: Wieso Angular 7
	//TODO: Wieso CI? Und Testing beschreiben //TODO: Testting pipeline bild
	//TODO: Hosting bei Heroku

#Deployment
	//TODO: Deploymentdiagramm bild einfügen


	\subsection{Zielsetzungen}
	\subsubsection{Muss-Ziele}
	\begin{enumerate}
		\item \faGlobe~   Globaler Gruppen Chat
		\item \faUser~    Hinterlegen eines Benutzernamens
		\item \faKey~     Benutzer muss den Chat später wieder aufnehmen können. z.B. Cookies oder Session Storage
		\item \faMobile~  Die Seite für Mobile geräte optimieren
	\end{enumerate}

	\subsubsection{Kann-Ziele}
	\begin{enumerate}
		\item \faUsers~   Private Chats zwischen zwei Personen
		\item \faGoogle~  Authentifizierung über Google Accounts, aber trotzdem anonym
	\end{enumerate}
\end{document}
