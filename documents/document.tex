\documentclass[12pt]{article}

\usepackage[utf8]{inputenc}
\usepackage[T1]{fontenc}

% clickable links in pdf
\usepackage{hyperref}

% make title variable available
\usepackage{titling}

% create own header footer
\usepackage{fancyhdr}

% images
\usepackage{graphicx}

% grafics powerfull and hard
\usepackage{tikz}

% character spacing to fill line
\usepackage{microtype}

% image placement with [H]
\usepackage{float}

% fancy quotes
\usepackage{epigraph}

% extra symbols
\usepackage{textcomp}

% display code
\usepackage{listings}

% make code copyable
\lstset{
upquote=true,
columns=fullflexible,
literate={*}{{\char42}}1
         {-}{{\char45}}1
}

\newsavebox{\picbox}

\graphicspath{ {./images} }

% command for rounded corners
\newcommand{\cutpic}[3]{
  \savebox{\picbox}{\includegraphics[width=#2]{#3}}
  \tikz\node [draw, rounded corners=#1, line width=4pt,
    color=white, minimum width=\wd\picbox,
    minimum height=\ht\picbox, path picture={
      \node at (path picture bounding box.center) {
        \usebox{\picbox}};
    }] {};}


\title{
  \Huge
  \textbf{Globalizer} \\
  \vspace{0.2cm}
  \LARGE
  Anforderungsspezifikationen
}

\date{25-10-2018}
\author{
  Jonas Koller \\
  Donato Wolfisberg
}

\pagestyle{fancy}
\fancyhf{}
\rhead{\thedate}
\lhead{Globalizer Anforderungsspezifikationen}
\rfoot{\thepage}
\lfoot{Jonas Koller \& Donato Wolfisberg}


\begin{document}
  \begin{titlepage}
    \pagenumbering{gobble}

    \begin{center}
      \vspace*{-2cm}
      \cutpic{0.8cm}{4cm}{./images/logo.jpg}

      \thetitle

      \vspace{2cm}

      \textbf{\theauthor}
    \end{center}

    \vfill

    \begin{figure}[H]
        \makebox[\linewidth]{
            \includegraphics[width=1.3\linewidth]{./images/globe.jpg}
        }
        \vspace*{-3cm}
    \end{figure}
  \end{titlepage}



  \newpage

  \section{Zweck des Dokuments}
    In diesem Dokument werden wir die Anforderungen an unser Projekt “Globalizer”.
    Das Projekt gilt im aktuellen Zustand als angenommen,
    da der Projektantrag durch Herr Manfred Rötheli angenommen wurde. Wir werden be\-schreiben,
    welche Anforderungen wir erfüllen möchten (“soll”-Anforderungen) und welche optionalen
    “kann”-Anforderungen wir eventuell auch erfüllen können.
    Es soll eine Übersicht für unser Vorhaben sein. Weiter sollen unsere Prozesse
    so gut wie möglich durch Grafiken dargestellt und visualisiert werden.
    Bei Fragen wenden sie sich an den Projektleiter (jonas.koller@gmx.ch).

  \section{Allgemeine Informationen}
    Hier folgt eine kurze Auflistung der Informationen zu diesem Dokument,
    dem Entwicklungsteam und dem aktuellen Stand.


  \pagenumbering{Roman}
  \tableofcontents
  \newpage

  \pagenumbering{arabic}

  \section{Montag}





\end{document}
